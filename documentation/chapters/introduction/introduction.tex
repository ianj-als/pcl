\chapter{Introduction}
Pipeline Creation Language (PCL) is a general purpose language for creating non-recurrent software pipelines. PCL is a small grammar for combining computation using re-usable components which can be shared or distributed. A number of combinator operators are defined which execute components sequentially or in parallel. Also, a number of pre-defined components can be used to ``glue'' components together, merge parallel outputs, or conditional execute components.

PCL was developed as part of the MosesCore project sponsored by the European Commission's Seventh Framework Programme (Grant Number 288487) \url{http://www.statmt.org/mosescore/}. For more information on the Seventh Framework Programme please see \url{http://cordis.europa.eu/fp7/home_en.html}.

\section{License and Availability}
PCL compiler and runtime has been released under a LGPL v3.0\footnote{See the GNU Lesser General Public License at \url{http://www.gnu.org/copyleft/lesser.html}} license. It is available from GitHub using the following command:
\begin{verbatim}
git clone https://github.com/ianj-als/pcl.git
\end{verbatim}
