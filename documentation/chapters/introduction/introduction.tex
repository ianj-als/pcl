\chapter{Introduction}
Pipeline Creation Language (PCL) is a general purpose language for creating non-recurrent software pipelines. PCL is a small grammar for combining computation using re-usable components which can be shared or distributed. A number of combinator operators are defined which execute components sequentially or in parallel. Also, a number of pre-defined components can be used to ``glue'' components together, merge parallel outputs, or conditionally execute components.

PCL was developed as part of the MosesCore project sponsored by the European Commission's Seventh Framework Programme (Grant Number 288487) \url{http://www.statmt.org/mosescore/}. For more information on the Seventh Framework Programme please see \url{http://cordis.europa.eu/fp7/home_en.html}.

\section{License and Availability}
PCL compiler and runtime has been released under a LGPL v3.0\footnote{See the GNU Lesser General Public License at \url{http://www.gnu.org/copyleft/lesser.html}} license. It is available from GitHub using the following command:
\begin{verbatim}
git clone https://github.com/ianj-als/pcl.git
\end{verbatim}

\section{Dependencies}
Today the PCL compiler and runtime scripts require two dependencies that should be installed manually. The first dependency is PLY: a parser library. PCL requires version 3.4 of PLY which can be found at \url{http://www.dabeaz.com/ply/ply-3.4.tar.gz}. Follow the instructions on how to build and install PLY v3.4. Addtionally, if you have Pyhton's \texttt{easy\_install} you can issue the command:
\begin{center}
\texttt{sudo easy\_install ply}
\end{center}

The second dependency is Pypeline, the Python pipelining library that underpins PCL. Pypeline is a submodule of your PCL git clone. If you haven't already initialised the submodule, use the following command now to do so:
\begin{center}
\texttt{git submodule update --init}
\end{center}
Pypeline can be installed using the instructions in the \texttt{README.md} in the directory \texttt{libs/pypeline}, or by setting your \texttt{PYTHONPATH} environment variable to:
\begin{center}
\texttt{<git clone root>/pcl/libs/pypeline/src}
\end{center}

Running the Python REPL you should now be able to import the PLY and Pypeline packages using the following Python commands:
\begin{verbatim}
$ python
Python 2.7.3 (default, Apr 10 2013, 06:20:15) 
[GCC 4.6.3] on linux2
Type "help", "copyright", "credits" or "license" for more
information.
>>> import ply
>>> import pypeline
>>> 
\end{verbatim}
